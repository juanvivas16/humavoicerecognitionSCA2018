\documentclass[12pt]{beamer}
\usepackage[utf8]{inputenc}
\usepackage[spanish]{babel}
\usepackage[document]{ragged2e}
\usetheme{Madrid}
\begin{document}
  \author{Juan Andr\'es Vivas}
  \title{Reconocimiento de voz humana a partir de un archivo de audio}
  \subtitle{Datos a usar y como capturarlos}
  %\logo{}
  %\institute{}
  %\date{}
  %\subject{}
  %\setbeamercovered{transparent}
  %\setbeamertemplate{navigation symbols}{}
  \begin{frame}[plain]
  \maketitle
\end{frame}

\begin{frame}
\frametitle{Datos a utilizar}
\justify{El conjunto de datos a utilizar serán archivos de audio con una duración de 2 segundos, formato WAV a una frecuencia de 44100 Hz.}
\end{frame}



\begin{frame}
\frametitle{Cantidad total requerida de datos}

\justify{La cantidad total aproximada de datos requeridos será de 1000 muestras, la cual se comprenderá en un 60\% de voz y un 40\% de ruidos diferentes a la voz humana entendible.}

\end{frame}


\begin{frame}
\frametitle{Cantidad de datos capturados}

\justify{La cantidad de datos capturados hasta el momento es de 35 muestras con ruido y 200 muestras de voz.}

\end{frame}


\begin{frame}
\frametitle{Consideraciones para capturar los datos}

\justify{

\begin{enumerate}
	\item Los datos de voz se capturan con un micrófono que aislé en la medida de lo posible el ruido externo.
	
	\item El ruido es capturado con un micrófono de celular.
	
	\item La duración de los archivos es variable, luego sera dividida a segmentos de 2 segundos y convertida al formato requerido.
	
	\item Se tomara como datos de ruido los habituales dentro de un salón o laboratorio, los cuales pueden ser murmullos, teclados de computadora, silencios, pasos, etc.

\end{enumerate}

}

\end{frame}


\begin{frame}
\frametitle{Pasos detallados para capturar los datos }
\justify{
	
	\begin{enumerate}
		\item Archivos de voz utilizados en los proyectos de Alvaro y Nerio.
		
		\item Se captura la voz usando un audífono con micrófono y un programa de grabación hecho en python.
		
		\item El ruido es capturado haciendo grabaciones con el micrófono del celular dentro del ambiente controlado, tratando de reproducir los sonidos deseados.
		
		
	\end{enumerate}
	
}

\end{frame}

\end{document}